\documentclass[12pt]{book}
\usepackage{listings, setspace, url}

\setlength{\parindent}{0pt}
\setlength{\parskip}{1ex}

\begin{document}

\lstset{language=sh, aboveskip={\bigskipamount}}

\title{{\Huge \bf Using oh}\thanks{Much of this document
shamelessly copied from Steve Bourne's ``An Introduction to the UNIX
Shell''.~\cite{sh}}}
\author{}
\date{}

\maketitle


\chapter{Introduction}

In addition to being a programming language, oh also provides a
command-line interface to Unix and Unix-like systems.


\section{Simple Commands}

Simple commands consist of one or more words separated by blanks.
The first word is the name of the command to be executed; any remaining
words are passed as arguments to the command.

\begin{lstlisting}
	ls -la
\end{lstlisting}

is a command that prints a list of files in the current directory.
The argument -l tells ls to print status information, size and the
creation date for each file.

Multiple commands may be written on the same line separated by a
semicolon.


\section{Input Output Redirection}

Standard input, standard output and standard error are initially
connected to the terminal. Standard output may be sent to a file.  

\begin{lstlisting}
	ls -l >file
\end{lstlisting}

The notation \verb%>%file is interpreted by the shell and is not
passed as an argument to ls. If the file does not exist then the
shell creates it, otherwise the original contents of the file are
replaced with the output from ls. Output may be appended to a file.

\begin{lstlisting}
	ls -l >>file
\end{lstlisting}

Standard error may be redirected

\begin{lstlisting}
	ls -l !>file
\end{lstlisting}

or appended to a file

\begin{lstlisting}
	ls -l !>>file
\end{lstlisting}

Standard input may also be redirected

\begin{lstlisting}
	wc -l <file
\end{lstlisting}


\section{Pipelines and filters}

The standard output of one command may be connected to the standard
input of another command using the pipe operator

\begin{lstlisting}
	ls | wc -l
\end{lstlisting}

The commands connected in this way constitute a pipeline. The
overall effect is the same as

\begin{lstlisting}
	ls >file; wc -l <file
\end{lstlisting}

except that no file is used. Instead the two processes are connected
by a pipe and are run in parallel.

A filter is a command that reads its standard input, transforms it
in some way, and prints the result as output. One such filter, grep,
selects from its input those lines that contain some specified
string. 

\begin{lstlisting}
    	ls | grep old
\end{lstlisting}

A pipeline may consist of more than two commands.

\begin{lstlisting}
    	ls | grep old | wc -l
\end{lstlisting}


\section{File name generation}

The oh shell provides a mechanism for generating a list of file
names that match a pattern.

\begin{lstlisting}
    	ls -l *.c
\end{lstlisting}

generates, as arguments to ls, all file names in the current
directory that end in .c. The character * is a pattern that will
match any string including the null string. In general 
patterns are specified as follows.

\begin{center}
\begin{tabular}{|c|l|}
\hline
\verb%*% & Matches any string of characters including the null string. \\
\verb%?% & Matches any single character. \\
\verb%[...]% & Matches any one of the characters enclosed. A pair separated \\
& by a minus will match a lexical range of characters. \\
\hline
\end{tabular}
\end{center}

For example,

\begin{lstlisting}
    	[a-z]*
\end{lstlisting}

matches all names in the current directory beginning with one of
the letters a through z.

\begin{lstlisting}
    	/usr/home/?
\end{lstlisting}

matches all names in the directory /usr/home that consist of a
single character. If no file name is found that matches the pattern
then the pattern is passed, unchanged, as an argument.

There is one exception to the general rules given for patterns.
The character . at the start of a file name must be explicitly
matched.

\begin{lstlisting}
    	echo *
\end{lstlisting}

will therefore echo all file names not beginning with a . in the
current directory.

\begin{lstlisting}
    	echo .*
\end{lstlisting}

will echo all those file names that begin with . as the . was
explicitly specified. This avoids inadvertent matching of the names
. and .. which mean the current directory and the parent directory
respectively. (Notice that ls, by default, suppresses information
for the files . and ..).


\section{Quoting} 

Characters that have a special meaning to the shell, such as \verb%<%
\verb%>% \verb%|% \verb%&%, are called metacharacters. Any character
preceded by a \ is excaped and loses its special meaning, if any. The
\verb%\% is elided so that

\begin{lstlisting}
    	echo \?
\end{lstlisting}

will echo a single \verb%?%, and

\begin{lstlisting}
    	echo \\
\end{lstlisting}

will echo a single \verb%\%. To allow long strings to be continued
over more than one line newlines can be excaped with a \verb%\%.

\verb%\% is convenient for escaping single characters. When more
than one character needs escaping the above mechanism is clumsy and
error prone. A string of characters may be quoted by enclosing the
string between double quotes.

\begin{lstlisting}
    	echo "xx****xx"
\end{lstlisting}

will echo

\begin{lstlisting}
    	xx****xx
\end{lstlisting}

The quoted string may not contain a double quote but may contain
newlines, which are preserved.


\chapter{Oh programming}

In addition to providing a command-line interface to Unix and
Unix-like systmes, oh is also a programming language.


\section{Variables}

In oh, variables are declared using the keyword 'define'. For example,

\begin{lstlisting}
	define x "hello"
\end{lstlisting}

Values are assumed to be strings. To create another type of value
use a generator.

\begin{lstlisting}
	define i: integer 0
\end{lstlisting}

The following types are available: boolean, float, integer, status,
string, symbol. The 'status' type is in integer that evaluates to
true in a boolean context when equal to zero and false otherwise.
The status type is useful for treating the return status from methods
and Unix commands in a uniform fashion.


\section{Lists}

Lists are a fundamental oh data type. The are modelled after Lisp's
list type. Like Lisp, oh uses the same syntax for code and data.
Many interactive commands are simple lists but more structured lists
are also possible. The command

\begin{lstlisting}
	define i: integer 0
\end{lstlisting}

could also be expressed as

\begin{lstlisting}
	define i (integer 0)
\end{lstlisting}

If the last element in a list is a sub-list a colon may be used in
place of parenstheses. If the last elements in a list are all
sub-lists curly braces may be used. The following

\begin{lstlisting}
	while line {
		write line
		set line: readline
	}
\end{lstlisting}

could also be expressed as

\begin{lstlisting}
	while line (write line) (set line: readline)
\end{lstlisting}

This small bit of syntactic sugar significantly reduces the number of
parentheses needed when writing oh scripts.


\section{Control Flow}

if, for, while


\section{Methods}

Methods can be created with 'method' TODO
Builtins can be created with 'builtin' TODO - globbing, no passing
of complex types.

\bibliographystyle{abbrv}
\bibliography{oh}

\end{document}

