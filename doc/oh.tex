\documentclass[12pt]{book}
\usepackage{listings, setspace, url}
\begin{document}

\title{\LARGE \bf Using oh}
\author{}
\date{}

\maketitle

\chapter{Introduction}
Oh is a freely available shell released under an MIT-style license that
retains the look and feel of current Unix shells while adding support for
higher-order, object-based and concurrent programming.

Oh was inspired by previous attempts to combine Lisp and the Unix shell
\cite{ALispShell, scsh}. Unlike previous attempts, however, oh is not
implemented by embedding a Unix shell in Lisp. Instead oh was designed
from scratch so that it could not only incorporate features from Lisp but
also modify and correct those characteristics that do not interact well
with the constraints imposed on oh as a command language.

Oh introduces a novel form of environment and exposes these environments
as first-class values. In doing so, oh is able to unify the treatment of
what is normally a number of limited and specialized mechanisms in addition
to supporting a powerful and flexible form of object-oriented programming.

Oh also exposes channels, which are implicit in other shells, as first-class
values. Along with this, oh introduces a unified format for the
representation of code and data. This allows data to be marshalled and
unmarshalled easily; shared between processes; and piped from, to or even
through external programs.

\bibliographystyle{abbrv}
\bibliography{oh}

\end{document}
